\usetikzlibrary{decorations.markings}
\usetikzlibrary{shapes.geometric} 
\usetikzlibrary{shapes.symbols} 
\def\Dcup{\Diagram{\draw (1,1) .. controls +(down:0.5) and +(down:0.5) .. (2,1);}}
\def\Dcap{\Diagram{\draw (1,0) .. controls +(up:0.5) and +(up:0.5) .. (2,0);}}
\def\Did{\Diagram{\draw (1,0) -- (1,1);}}
\def\Dcross{\Diagram{\draw (0,0) -- (1,1);\draw (1,0) -- (0,1);}}

\def\Rcap{\Diagram{\draw[mid arrow] (1,0) .. controls +(up:0.5) and +(up:0.5) .. (2,0);}}
\def\Lcap{\Diagram{\draw[mid arrow] (2,0) .. controls +(up:0.5) and +(up:0.5) .. (1,0);}}
\def\Fid{\Diagram{\draw[mid arrow] (1,0) -- (1,1);}}
\def\Bid{\Diagram{\draw[mid arrow] (1,1) -- (1,0);}}

\def\textchoice#1#2#3#4#5{\ifmmode\mathchoice{#1}{#2}{#3}{#4}\else #5\fi}

\def\Diagram#1{%
\thinspace\textchoice
{\DiagramA{0.25em}{1em}{1em}{#1}}
{\DiagramA{0.25em}{0.7em}{1em}{#1}}
{\DiagramA{0.175em}{0.55em}{0.7em}{#1}}
{\DiagramA{0.125em}{0.4em}{0.5em}{#1}}
{\DiagramA{0.25em}{0.7em}{1em}{#1}}%
\thinspace%
}

\def\DiagramA#1#2#3#4{%
%\expandafter\tikzpicture\Gparam
\tikzpicture[
	%arrow styles
	mid arrow/.style={postaction={decorate,decoration={
        markings,
        mark=at position .5 with {\arrow{>}}
      }}},
	late arrow/.style={postaction={decorate,decoration={
        markings,
        mark=at position .8 with {\arrow{>}}
      }}},
	%point styles
	circ/.style={circle,draw,fill=white,inner sep=1pt},
	bcirc/.style={circle,draw,fill=black,inner sep=1pt},
	map/.style={draw,fill=white,trapezium,trapezium left angle=90,trapezium right angle=120,inner sep=1pt},
	mapC/.style={draw,fill=white,trapezium,trapezium left angle=120,trapezium right angle=90,inner sep=1pt},
	mapT/.style={draw,fill=white,trapezium,trapezium left angle=60,trapezium right angle=90,inner sep=1pt},
	mapA/.style={draw,fill=white,trapezium,trapezium left angle=90,trapezium right angle=60,inner sep=1pt},
	selfC/.style={draw,fill=white,trapezium,trapezium left angle=120,trapezium right angle=120,inner sep=1pt},
	selfCR/.style={draw,fill=white,trapezium,trapezium left angle=60,trapezium right angle=60,inner sep=1pt},
	selfT/.style={draw,fill=white,trapezium,trapezium left angle=60,trapezium right angle=120,inner sep=1pt},
	selfTR/.style={draw,fill=white,trapezium,trapezium left angle=120,trapezium right angle=60,inner sep=1pt},
	selfA/.style={draw,fill=white,signal,signal to=east,inner sep=1pt},
	selfAR/.style={draw,fill=white,signal,signal to=west,inner sep=1pt},
	square/.style={draw,fill=white,rectangle,inner sep=1pt},
	vec/.style={draw,fill=white,isosceles triangle, shape border rotate=-90,isosceles triangle apex angle=60,inner sep=1pt},
	covec/.style={draw,fill=white,isosceles triangle, shape border rotate=90,isosceles triangle apex angle=60,inner sep=1pt},
]
\pgfsetbaseline{#1}
\pgfsetxvec{\pgfpoint{#2}{0em}}
\pgfsetyvec{\pgfpoint{0em}{#3}}
\pgfsetlinewidth{\Dlinewidth em}
#4
\endtikzpicture
}

\def\Dlinewidth{0.05}

\def\DstringsB #1/#2(#3,#4){
\ifnum#1>0
\pgfmathparse{#3-(#1-1)/2}
\let\x=\pgfmathresult
\pgfmathparse{#3+(#1-1)/2}
\let\xx=\pgfmathresult
\foreach \a in {\x,...,\xx}
	\draw (\a,#4-\Dyshift) .. controls +(up:0.25) .. (#3,#4);
\fi
\ifnum#2>0
\pgfmathparse{#3-(#2-1)/2}
\let\x=\pgfmathresult
\pgfmathparse{#3+(#2-1)/2}
\let\xx=\pgfmathresult
\foreach \a in {\x,...,\xx}
	\draw (\a,#4+\Dyshift) .. controls +(down:0.25) .. (#3,#4);
\fi
}

\newcount\tmpcount
\def\Dcount#1{\ifx/#1/\else\advance\tmpcount by1\fihere\Dcount\fi}

\def\DstringsC [#1/#2](#3,#4){
\tmpcount=-1
\Dcount#1//
\pgfmathparse{#3-\the\tmpcount/2}
\let\x=\pgfmathresult
\scope[
	up arrow/.style={postaction={decorate,decoration={
        markings,
        mark=at position .3 with {\arrow{>}}
      }}},
	down arrow/.style={postaction={decorate,decoration={
        markings,
        mark=at position .3 with {\arrow{<}}
      }}},
]
\DstringsD{#3}{#4}#1//
\endscope
\tmpcount=-1
\Dcount#2//
\pgfmathparse{#3-\the\tmpcount/2}
\let\x=\pgfmathresult
\scope[
	down arrow/.style={postaction={decorate,decoration={
        markings,
        mark=at position .7 with {\arrow{>}}
      }}},
	up arrow/.style={postaction={decorate,decoration={
        markings,
        mark=at position .7 with {\arrow{<}}
      }}},
]
\DstringsE{#3}{#4}#2//
\endscope
}

\def\fihere#1\fi{\fi#1}
\def\DstringsD#1#2#3{\ifx/#3/\else
	\csname D#3\endcsname (\x,#2-\Dyshift) .. controls +(up:0.25) .. (#1,#2);
	\pgfmathparse{\x+1}
	\let\x=\pgfmathresult
	\fihere\DstringsD{#1}{#2}\fi
}
\def\DstringsE#1#2#3{\ifx/#3/\else
	\csname D#3\endcsname (\x,#2+\Dyshift) .. controls +(down:0.25) .. (#1,#2);
	\pgfmathparse{\x+1}
	\let\x=\pgfmathresult
	\fihere\DstringsE{#1}{#2}\fi
}

\def\Dstrings{
\futurelet\nextchar\DstringsA
}

\def\DstringsA{\ifx[\nextchar\let\next=\DstringsC\else\let\next=\DstringsB\fi\next}

\def\Dmor#1#2 (#3,#4){
\def\Dyshift{0.5}
\ifx;#2;\else\Dstrings#2(#3,#4)\fi
\node at (#3,#4) [#1] {};
}

\def\DMor#1#2 (#3,#4)#5{
\def\Dyshift{1}
\ifx;#2;\else\Dstrings#2(#3,#4)\fi
\node at (#3,#4) [#1] {#5};
}

\def\spider{\futurelet\nextchar\spiderA}
\def\spiderA{\ifx[\nextchar\def\next[##1]{\Diagram{\Dmor{bcirc}[##1] (1,0.5)}}\else\def\next##1{\Diagram{\Dmor{bcirc}##1 (1,0.5)}}\fi\next}
\def\wspider{\futurelet\nextchar\wspiderA}
\def\wspiderA{\ifx[\nextchar\def\next[##1]{\Diagram{\Dmor{circ}[##1] (1,0.5)}}\else\def\next##1{\Diagram{\Dmor{circ}##1 (1,0.5)}}\fi\next}

\def\Ddef#1#2{\expandafter\def\csname D#1\endcsname{#2}}
\Ddef>{\draw[<-]}
\Ddef<{\draw[>-]}
\Ddef-{\draw}
\expandafter\def\csname D0\endcsname#1;{}
\expandafter\def\csname D.\endcsname (#1,#2)#3;{\node at (#1,#2) {$\dots$};}

\endinput
